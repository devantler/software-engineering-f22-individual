\section{Case study relevance}

The case study helped the group better understand how to use UPPAAL for model checking. However, as all five timed automata diagrams were available in the appendix of \cite{gearcontrol}, the task was made simple by only requiring us to recreate the templates in UPPAAL. Recreating the templates only proved slightly tricky as the diagrams were not directly translatable into UPPAAL due to differences in notation. 
Overall, the task was very time-consuming, but the learning potential was not as significant. Internally in the group, we discussed that a smaller task that requires us to construct all elements of the UPPAAL system could potentially have more substantial learning potential.

During the implementation of the case study into UPPAAL, techniques were obtained such as: splitting up templates; working with invariants; implementing urgent and committed locations on nodes; working with updates, guards, and syncs on transitions; and writing verification rules. Converting a real system into a UPPAAL system was not learned as the UPPAAL system was recreated from the available timed automata diagrams.