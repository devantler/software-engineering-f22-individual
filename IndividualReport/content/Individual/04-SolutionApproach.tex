\subsection{Solution Approach}

The solution to the problem is to (1) support defining Unit Tests and UPPAAL Queries in the \acrshort{dsl}. (2) Code-generate Unit Tests to the Orchestrator, and code-generate UPPAAL Queries to the UPPAAL Project. (3) Ensure Unit Tests are run as part of the deployment pipeline.\\

The solution disregards the system's inability to auto-pull and start docker images on the Raspberry Pi, as the Pi is no longer available. A simple solution for this could have been to add a crontab job to pull the latest image from GitHub Packages and restart the Docker container. A more robust solution would be to use a Container Orchestrator to manage the Docker containers e.g. Docker Swarm, Nomad HashiCorp, or Kubernetes.\\

The solution also disregards the system being a local system. Connecting to services remotely is impossible when the services are not open to the public. A workaround is to use alternative remote services to communicate with the Orchestrator during the deployment pipeline.\\

\subsubsection{Advantages and Disadvantages}
The solution provides a set of advantages to the system:

\begin{itemize}
    \item Making the system testable by end-users.
    \item Making model-checking for future issues and requirements possible.
    \item Ensuring the correctness of the Orchestrator before deployment.
    \item Ensuring remote services are available.
    \item Ensuring connectivity to remote service is possible.
\end{itemize}

However, it still has a number of disadvantages:

\begin{itemize}
    \item Health checks in Unit Tests only ensure service is available at one given time.
    \item Defining Unit Tests in the \acrshort{dsl} is limited by scope.
    \item Defining UPPAAL Queries in the \acrshort{dsl} is limited by scope.
\end{itemize}
