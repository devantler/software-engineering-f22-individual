\subsection{Conclusion}

To solve the problem and objective stated in \autoref{sec:problem-objective}, a prototypical assembly line system has been built. It can move, store and scan items, with three separate devices, that are interconnected utilizing \acrshort{mqtt}. Items put on a disk is automatically moved to a camera, that can scan items, so they can be picked up by a crane and be put into a storage container matching the scanned colour. The system operates both synchronously and asynchronously, which means every device in the system is capable of executing tasks individually, but they await each other in cases where necessary. This ensures that, for example, the disk waits for the crane to have picked up an item from the disk, before turning itself around to the next item.\\

The problem and objective have been satisfied by the following achievements:

\paragraph{1.} An external \acrshort{dsl} has been successfully developed in Xtext (consisting of both configuration and logic) to create a simple and natural language that non-programmers can use to program their assembly line. Verification and scoping rules have been added to the \acrshort{dsl} to ensure that the end-users cannot write something that is not allowed. All necessary artefacts in C\# have been code-generated in Xtend, which makes the process of creating new programs that the system understands automatic.

\paragraph{2. }A Raspberry Pi has been included to enable the orchestrator to use Wi-Fi to send commands through \acrshort{mqtt}-topics across all devices. All hardware in the system has been connected to two different ESP32s to make sure that all parts of the system can respond correctly to the commands sent by the orchestrator.

\paragraph{3.} The logging level is remotely dynamically changeable for the devices, in a way where they can save on bandwidth if needed by lowering the logging level. It is stored in a database and can be searched and shown as needed.

\paragraph{4.} Model checking has been used through UPPAAL-templates that cover all the processes within the system. All templates have been code-generated by the \acrshort{dsl}. This means that both the C\# files and the UPPAAL-templates are all being generated by the same program, automatically. Queries have been written to verify and validate the requirements. The results indicate that the system mostly behaves reliably and safely.\\
